\chapter{Analysis and Hyphothesis}
\label{ch:analysis_and_hypothesis}

This chapter summarises on the findings of the literature review and presents the motivation for a new change-based persistence format and a novel approach to improve the performance of model differencing and conflict detection by exploiting the change-based persistence. Based on the findings in Chapter \ref{ch:literature_review}, this chapter presents the hypothesis and research questions addressed in this study. It also presents an overview of the employed research method in answering the research questions. 

\section{Summary of Findings}
\label{sec:a_new_change_based_persistence}

Performing model differencing and conflict detection in state-based persistence can be expensive in computation time \cite{DBLP:conf/edoc/KoegelHLHD10}. This is due to the state-based model differencing that requires every element of two versions being compared to be inspected, matched, and diffed in order to identify their differences \cite{emfcompare2018developer}. Even persisting state-based models using database backends -- such as in Teneo \cite{eclipse2017teneo}, CDO \cite{eclipse2019cdo}, Morsa \cite{DBLP:conf/models/Espinazo-PaganCM11}, and NeoEMF \cite{daniel2016neoemf} -- can only reduce the overhead cost of loading models, since all elements still need to be checked. Imagine if we have only made small changes on a model, but all of its elements have to be examined to identify differences. This approach is not efficient and can cause bottlenecks, especially in collaborative environments where models are often managed in different concurrent versions; differencing, conflict detection, and merging are common in this context.

As an alternative to state-based persistence, change-based persistence has the potential to deliver high-performance model differencing and conflict detection since the change history of a model is already contained in the model's change-based representation  \cite{DBLP:conf/sde/LippeO92,DBLP:conf/caise/IgnatN05,koegel2010emfstore}, and therefore deriving changes through model differencing is not required as in state-based persistence. Moreover, model differencing and conflict detection in change-based persistence can also be more accurate than performing them in state-based persistence since detailed information about the changes is also contained in the persistent representation, such as order of changes, types of changes, and elements affected by changes  \cite{DBLP:journals/entcs/RobbesL07,DBLP:conf/sde/LippeO92,DBLP:conf/caise/IgnatN05,mens2002state}.  

So far, we can only identify EMF Store as the only implementation of change-based model persistence that conforms to Eclipse Modeling Framework (EMF). Nevertheless, this research does not consider to use and extend EMF Store due to several considerations. Firstly, EMF Store is a full-fledged client-server model repository and versioning system which means it demands a certain degree of administration activities (e.g. server configuration, user authentication and authorisation) and creates dependency on EMF Store. We favour avoiding such administration activities and dependency and prefer a solution that can version on share models through different text-oriented version controls (e.g. SVN, Git). Secondly, it does not scale due to performance degradation as more models/users are added to a repository and models grow in sizes \cite{KolovosRMPGCLRV13} (we have evaluated the latter ourselves in Sections \ref{sec:evaluation_6} and \ref{sec:evaluation_discussion}). Thirdly, EMF Store does detect conflicts between changes that produce different states when merging. However, it cannot be used directly for model differencing. It is not designed to identify differences between two versions of a model. Fourthly, since it works purely in changes, and it does not consider eventual states of models in detecting conflicts \cite{DBLP:conf/sfm/BroschKLSWW12}. In consequence, if an element has been changed concurrently, but the changes produce eventual states that are equal to their original state, EMF Store still treats these changes as they are in conflict. Lastly, EMF Store has been in maintenance mode, no active feature development going on, and might be declared end-of-life in the year 2022 \cite{emfstore2019what}.

Based on these considerations, we aim for a new change-based persistence for EMF-based models. The implementation should be able to capture and persist all the necessary changes of models into text-based files, and it should be able to exploit the persisted changes to produce high-performance model differencing and conflict detection.

\section{Hypothesis and Research Questions}
\label{sec:research_questions}
The research presented in this thesis aims at improving the performance of model differencing and conflict detection. Based on the literature review, change-based model persistence has the potential to deliver such performance. In order to assess whether that change-based persistence can be used to improve the performance of model differencing and conflict detection, the following hypothesis has been established \textbf{``a textual change-based model persistence approach can outperform existing model persistence formats in terms of model saving, model differencing and conflict detection time, with an overhead in terms of model loading time and memory use''}. 

In this thesis, the word 'model' refers to typed object graphs that conform to 3-layer object-oriented metamodelling architectures such as Eclipse Modelling Framework (EMF) \cite{eclipse2019emf}. 

Model differencing is used to identify the differences between versions of a model. For example, determining what has been changed from an original version of a model, or comparing versions of a model created by different teams working independently. The main goal of conflict detection is to ascertain whether independent updates can be merged, or whether there are conflicts that need to be resolved first: elements or features that differ in ways that are incompatible. 

The execution time referred to in the hypothesis is the time required to perform model saving, model differencing, or model conflict detection. We are parcicularly interested in the benefits and the challenges of using change-based persistence for large models; these are models having more than a million elements as per \cite{daniel2016neoemf,DBLP:conf/models/Espinazo-PaganCM11}. Model loading time is the amount of time required to load a model from its persistent representation into memory. Memory use is the size of the memory occupied during model saving, loading, differencing, and conflict detection.

To assess the validity of the hypothesis, this work aims to answer the following research questions: 
\begin{enumerate} 
  \item \textbf{How can models be persisted in a change-based format, and how does change-based persistence perform, compared to state-based persistence in terms of loading and saving models? (RQ1)} 
  
  The concept of change-based persistence has to be translated into an implementation in a modelling framework context so that it can be applied for model persistence, and therefore its impact on model loading and saving, and later model differencing, and model conflict detection can be assessed.
  
  \item \textbf{In a change-based format, how can differences between models be identified, and how does change-based model differencing perform, in terms of speed and memory footprint, compared to state-based model differencing? (RQ2)} 
  
  One of the main motivations for exploring of using change-based persistence in this work is to speed up model differencing. Due to the nature of change-based persistence, the mechanism to perform change-based model differencing will differ substantially from the current state-based model differencing approaches. It is expected that model differencing in change-based persistence will perform faster than model differencing in state-based persistence.        
  
  \item \textbf{Following change-based model differencing, how can conflicts be detected between versions of a model, and  how does change-based conflict detection perform, in terms of speed and memory, compared to state-based model conflict detection? (RQ3)} 
  
  The knock-on effect of change-based persistence on model conflict detection will also be investigated. It is expected that conflict detection of change-based models will be significantly faster than conflict detection of state-based models.
\end{enumerate}

\section{Research Method}
\label{sec:research_method}
In performing this research, this work follows the experimental process proposed by Wohlin et al. \cite{DBLP:books/daglib/0029933/Wohlin}. The experimental process consist of 5 activities: scoping, planning, operation, analysis and interpretation, and presentation and package.

\textbf{Scoping}. In the scoping activity, the hypothesis, goals, and objectives of an experiment have to be defined clearly \cite{DBLP:books/daglib/0029933/Wohlin}. Basili et al. \cite{basili1988tame} provide the following questions (scoping points) in their framework to help determining the scope of an experiment in software engineering: (1) what is studied? (object of study), (2) what is the intention? (purpose), (3) which effect is studied? (quality focus), (4) whose view? (perspective), and where is the study conducted? (context).

\textbf{Planning}. In the planning activity, these components have to be defined in detail: context selection, hypothesis formulation, variables selection, selection of subjects, experiment design selection, instrumentation, and validity evaluation\cite{DBLP:books/daglib/0029933/Wohlin}. The context can be offline vs. online, student vs. professional, toy vs. real problems, specific vs. general. Null and alternative hypotheses have to be stated formally, and the data gathered throughout the experiment should be used -- using appropriate statistical tests -- to reject the null hypothesis if possible. In the variables selection, the independent and dependent variables to be measured are determined. The subjects should be selected carefully that they are representative of the case experimented to generalise the results of the experiment. The experiment should be designed carefully to get the desired results, and suitable standard design types should be selected. Experiment objects, guidelines, and measurement instruments also should be defined to ensure the experiment is executable. Lastly, validity threats should be identified and evaluated.

\textbf{Operation}. The operation activity consists of three steps: preparation, execution, and validation \cite{DBLP:books/daglib/0029933/Wohlin}. 
In the preparation, all the required materials to facilitate the execution of the experiment are selected and prepared. The experiment can be executed in several ways, such as in one or multiple occasions, or one or multiyear.  While performing, the experiment has to be made sure if it is on the right track, not interrupted, and running correctly. The produced data also have to be validated if they are reasonable and collected orderly.

\textbf{Analysis and interpretation}.
To understand the data gathered, descriptive statistics and visualization can be used. Unnecessary data and variables also can be removed to facilitate analysis and interpretation. Hypothesis testing can be performed to reject or accept the experiment's hypotheses. The analysis and interpretation should explain how the data gathered contribute to the rejection or acceptance of the hypotheses. The results might be statistically insignificant, but the lessons might still worth to be learned \cite{DBLP:books/daglib/0029933/Wohlin}. 

\textbf{Presentation and package}. In this activity, experiment results should be documented and published in research papers for the dissemination of the results. The experiment also should be packaged to support other parties to replicate it \cite{DBLP:books/daglib/0029933/Wohlin}. 

\section{Conclusions}
\label{sec:conclusions_2b}
In this chapter, reflecting on the identified advantages, downsides, and challenges of current approaches on model persistence, differencing, and conflict detection in the literature review, we have pointed out some design considerations that the solution proposed in this research should deliver to achieve high-performance model differencing and conflict detection. From there, we established the hypothesis and research questions of this study. Finally, we presented the overview of the research method employed in this research. 