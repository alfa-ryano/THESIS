\chapter{Literature Review}
\label{ch:literature_review}

This chapter presents the literature review of this study. Firstly, it discusses some work on model persistence by highlighting some key characteristics -- unique features, strong points, and downsides -- of some existing implementations. It then summarises the advantages and drawbacks of the two main types of persistence: state-based and change-based persistence, and introduces desirable characteristics for a new change-based persistence implementation. This chapter then reviews the related work on identifying differences and detecting conflicts between versions of models. It then presents the current challenges that model differencing and conflict detection are dealing with as well as the downsides of existing approaches in tackling the problems, which gives motivation to this research to come up with a new solution. Thirdly, this chapter provides an overview of the research method employed in this research. Lastly, the conclusions of the literature review are presented.  

%\section{Introduction}
%\label{sec:introduction_2}
%The research introduced in this thesis aims at enabling high-performance model differencing and conflict detection through change-based persistence. This goal can only be delivered if this study can identify the strong points and downsides of current approaches of model persistence, differencing, and conflict detection and incorporate and address them in the solution of this research. Knowing the strong points and downsides is also helpful to find rooms for improvement and to deliver novelties in the solution. A literature review needs to be conducted in order to identify the strong points and downsides. The literature review of this study is presented in this chapter.
%
%Once the strong points, downsides, rooms for improvements and novelties have been gathered, they all need to be elaborated into a design of a solution. The solution needs to be translated into a working implementation, experimented, and evaluated to answer the research questions of this study. This sequence of work has to be guided by a research method to ensure it has been carried out properly. The research method employed in this study is also presented in this chapter. 
%
%The rest of this chapter is structured as follows. Section \ref{sec:model_persistence} reviews some key implementations of model persistence; their strong points and drawbacks. Section \ref{sec:model_differencing_and_conflict_detection} presents the current approaches in model differencing and conflict detection, their existing challenges, as well as the motivation for a new change-based model persistence. Section \ref{sec:benefits_and_novel_capabilities} presents briefly the research method employed in this research. Section \ref{sec:conclusions_2} concludes this chapter.

\section{Models in Model-Driven Engineering}
\label{sec:models_in_model_driven_engineering))}

First of all you should explain what kind of models this thesis is concerned with i.e. models conforming to object-oriented languages defined using technologies such as EMF and MOF

modelling tools that support the 3-layer metamodelling architectures of Eclipse Modeling Framework (EMF) \cite{steinberg2008emf}


\section{Model Persistence}
\label{sec:model_persistence}
In constructing models, modelling tools should be able to support model persistence so that models that are under construction can be saved at any time and reloaded for further modification. Majority of the tools persist models in a state-based format that is capturing a snapshot of a model at a time and then persist its entire state into storage. The model state can be persisted in different forms, such as text files, relational databases, or NoSQL databases.

\subsection{Text Files}
\label{sec:text_file}
The simplest and most common way to save a model is to persist it into a text file. By default, modelling tools that support the 3-layer metamodelling architectures of Eclipse Modeling Framework (EMF) \cite{steinberg2008emf} persist a model in a text file with a format of Metadata Interchange (XMI) -- a standard issued by Object Management Group (OMG) for exchanging metadata information via Extensible Markup Language (XML) \cite{omg2018xmi}. 

Since it is the default standard for persisting EMF models, it is supported by  most modelling tools. In order to modify a model persisted in an XMI file, such as performing create, read, update, delete (CRUD) operations, a tool has to de-serialise and load the model from the file into memory. This can be a problem when we only want to make a small number of changes but the size of the model is very large -- it takes considerable time and memory to load the model. Also, when saving, the model has to be persisted in its entirety causing inefficiency when we only made small number of changes. Since it is a text-based file, the model can be duplicated and shared with minimum effort, e.g. through manual copy or version control systems (e.g. Git \cite{git2019about}, SVN \cite{apache2019svn}). However, for model differencing (see Section \ref{sec:model_differencing_and_conflict_detection}), text-based differencing \cite{DBLP:journals/algorithmica/Meyers86} cannot be applied accurately to XMI files since essentially they are tree documents which require different differencing approaches \cite{wang2003xdiff}.

\subsection{Relational Databases}
\label{sec:relational_databases}
Models can also be persisted into relational databases. EMF Teneo \cite{eclipse2017teneo} is a solution that integerates EMF with existing persistency solutions, such as Hiberbate \cite{hibernate2019hibernateorm} and EclipseLink \cite{eclipse2019eclipselink}. Thus it can persists EMF models into relational database backends. In this way, EMF Teneo can utilise the power of storage, caching, and querying of the database backends. It also supports the automatic mapping of models to relational model schema with flexible mapping customisation. Using relational databases as its backends enables EMF Teneo to support lazy loading of models. So, when performing CRUD operations, it only loads and saves relevant elements and features -- not the entire model -- into and from memory, which is efficient in terms of memory usage.

Similar to EMF Teneo, Connected Data Objects (CDO) \cite{eclipse2019cdo} also supports persisting models into various database-backends model persistence (e.g. relational and NoSQL databases). It is a development-time model and metamodel repository as well as a distribution and runtime persistence framework for EMF-based application systems. It supports model versioning and can perform model differencing and conflict detection -- it uses EMF Compare \cite{emfcompare2018developer} to perform the comparison \cite{cdo2019emfcompare}. One downside of CDO is its adoption necessitates the use of a separate version control system (e.g. a Git repository for code and a CDO repository for models), which introduces fragmentation and administration challenges \cite{barmpis2014evaluation}.

\subsection{NoSQL Databases}
\label{sec:NoSQL Databases}
In the era where data is abundant and models are getting larger in sizes, the ability to handle large models is necessary. Tools, such as Morsa \cite{DBLP:conf/models/Espinazo-PaganCM11} and NeoEMF \cite{daniel2016neoemf}, has been developed to persist models into non-relational (NoSQL) databases. Morsa saves models in documents with MongoDB as its backend \cite{mongodb}, while NeoEMF persists models in multi, NoSQL backends: Neo4j \cite{neo4j2019neo4j} for Graph, MapDB \cite{mapdb2019mapdb} for Map, and Apache HBase \cite{apache2019hbase} for Column datastores. The advantages of using NoSQL databases are that users are given options to choose which datastores -- wih some degree for configuration -- that best fit with the characteristics of their models and metamodels and to maximise the features  the backends provide, such as lazy loading, caching, etc. Neither Morsa nor NeoEMF provide built-in support for versioning and models are eventually stored in binary files/folders which are known to be a poor fit for text-oriented version control systems like Git and SVN. 

\subsection{Change-based Representation}
\label{sec:change_based_representation}
All the solutions previously mentioned persist models in state-based format. EMF Store \cite{koegel2010emfstore} takes a different approach; it persists models in a change-based representation. EMF Store appears to be the only current implementation of change-based persistence for EMF models.

EMF Store is a model repository and supports collaborative editing and versioning of models \cite{emfstore2019what}. Instead of using standard text-oriented version controls (e.g., Git, SVN) for model versioning, EMF Store has its own dedicated, change-based, model-oriented versioning mechanism. Models are shared through a server and distributed to client applications. Clients can modify the models in parallel, offline or online, and synchronize with the server. Conflicts caused by concurrent modification are detected automatically and can be resolved interactively by users. The historical changes of models are kept on the server, and different versions of a model as well as changes that produce the them can be retrieved from the server. 

In EMF Store, to version models, a project has to be created first. A project can contain one or more models. Every project has a version history, and each version represents a commit of a client. A commit sends a package of changes to the server. The package itself contains a collection of operations applied to a version that transforms it to a newer version or can be said as the deltas between the two versions. The operations can be add, delete, set, unset or move that modifies an element or feature, or they can be a composite operation, an operation that consists of many operations, e.g. refactoring that moves a method to a superclass.

To obtain a specific version of an existing project, a client can perform a checkout. This version is called the base version on the client side. The client then can perform modification to this version. Every operation applied to the version is recorded by EMF Store. When the client commits, these operations are put into one package and sent to the server. If the base version is still the head version of the project on the server, the commit is accepted, and a new version is created. If they are different, it means that there is another client that already committed its changes to the server. Thus the current client has to synchronise it by updating its local project. This is the state where conflicts can happen potentially between the incoming and local changes, that is when they modify the same element or feature of a model. EMF Store performs conflict detection to identify conflicts automatically. The mechanism of EMF Store to identify conflicts is discussed in detail in Section \ref{sec:emfstore_conflict_detection}.

\begin{table*}[]
  \centering
  \caption{Advantages and downsides of different model persistence solutions.}
  \label{table:model_persistence_comparison}
\begin{scriptsize}
  \begin{tabular}{
      |>{\centering\arraybackslash}m{0.1\linewidth}
      |>{\centering\arraybackslash}m{0.4\linewidth}
      |>{\centering\arraybackslash}m{0.4\linewidth}
      |}
    \hline
    \textbf{Products} & \textbf{Advantages} & \textbf{Downsides} \\
    \hline
    XMI 
    &
    \begin{minipage}[t]{\linewidth}
      \raggedright
      \begin{itemize}[leftmargin=7pt]
        \setlength
        \item[+] default standard, widely supported
        \item[+] easy to be duplicated and shared by manual copy or text-oriented version controls
      \end{itemize}
    \end{minipage}
    & 
    \begin{minipage}[t]{\linewidth}
      \raggedright
      \begin{itemize}[leftmargin=5pt]
        \setlength
        \item[--] requires loading the entire model to modify
        \item[--] a model is saved in its entirety
        \item[--] supports text-oriented version controls, but applying text-based differencing might produce inaccurate results    
      \end{itemize}
    \end{minipage}
    \\
    \hline
    Teneo 
    & 
    \begin{minipage}[t]{\linewidth}
      \raggedright
      \begin{itemize}[leftmargin=7pt]
        \setlength
        \item[+] supports lazy loading, only load and save affected elements and features when performing CRUD operations
        \item[+] has the capabilities supported by database backends: rollback, caching, etc. 
      \end{itemize}
    \end{minipage}
    &
    \begin{minipage}[t]{\linewidth}
      \raggedright
      \begin{itemize}[leftmargin=7pt]
        \setlength
        \item[--] does not support model versioning, comparison, and merging
        \item[--] multiple concurrent accesses can cause a bottleneck
%        \item[--] performance on loading and saving for CRUD operations depends on the database backend
        \item[--] poor fit for text-oriented version controls since models are persisted in database
      \end{itemize}
    \end{minipage} 
    \\
    \hline
    CDO
    &
    \begin{minipage}[t]{\linewidth}
      \raggedright
      \begin{itemize}[leftmargin=7pt]
        \setlength
        \item[+] supports lazy loading, only load and save affected elements and features when performing CRUD operations
        \item[+] supports model versioning, comparison, and merging
        \item[+] has the capabilities supported by database backends: rollback, caching, etc. 
      \end{itemize}
    \end{minipage}
    &
    \begin{minipage}[t]{\linewidth}
      \raggedright
      \begin{itemize}[leftmargin=7pt]
        \setlength
%        \item[--] performance on loading and saving for CRUD operations depends on the database backends
        \item[--] fragmentation and administration challenges due to separation of version controls between models and code
        \item[--] poor fit for text-oriented version controls since models are persisted in database
      \end{itemize}
    \end{minipage} 
    \\
    \hline
    Morsa \& NeoEMF 
    & 
    \begin{minipage}[t]{\linewidth}
      \raggedright
      \begin{itemize}[leftmargin=7pt]
        \setlength
        \item[+] supports lazy loading, only load and save affected elements and features when performing CRUD operations
        \item[+] has the capabilities supported by NoSQL backends: handling big data, graph data, etc.
      \end{itemize}
    \end{minipage}
    &
    \begin{minipage}[t]{\linewidth}
      \raggedright
      \begin{itemize}[leftmargin=7pt]
        \setlength
        \item[--] does not support model versioning, comparison, and merging
%        \item[--] performance on loading and saving for CRUD operations depends on the database backends
        \item[--] poor fit for text-oriented version controls since models are persisted in database
      \end{itemize}
    \end{minipage} 
    \\
    \hline
    EMF Store
    &
    \begin{minipage}[t]{\linewidth}
      \raggedright
      \begin{itemize}[leftmargin=7pt]
        \setlength
        \item[+] supports semantic versioning of models that allows model merging and conflict detection to be more effective
      \end{itemize}
    \end{minipage}
    &
    \begin{minipage}[t]{\linewidth}
      \raggedright
      \begin{itemize}[leftmargin=7pt]
        \setlength
        \item[--] requires loading the entire model to modify
        \item[--] a model is saved in its entirety even for small changes
        \item[--] persists models in the forms of files/folders and using its own mechanism for model versioning thus poor fit for text-oriented version controls
      \end{itemize}
    \end{minipage} 
    \\
    \hline
  \end{tabular}
\end{scriptsize}
\end{table*}
  
The primary motivation EMF Store applies change-based approach is that calculating the differences between two versions in state-based persistence can be expensive and less accurate \cite{emfstore2019versioning} (state-based model differencing identify differences using LCS algorithms \cite{emfcompare2018developer,DBLP:journals/algorithmica/Meyers86}, not from the real changes). Since it follows a change-based approach, it does not store the state of every version. It only saves operations of each version in an ordered manner so that they can be executed and reversed to obtain the states between versions. Nevertheless, it also stores the intermediate cached states for selected versions, including the head version, to speed up the retrieval of specific versions.

The advantages of EMF Store are that it was designed to allow semantic versioning of models, which can support model differencing and conflict detection to be more accurate and efficient rather than performing them on state-based model persistence \cite{emfstore2019versioning}. By default, the packages of operations are persisted in XMI files but it can also be configured to use other backends, e.g. MongoDB \cite{emfstore2019mongodb}. The downsides of EMF Store are it has its own mechanism for controlling versions thus limits its adopters to use common text-oriented version controls \cite{emfstore2019getting}, such as Git and SVN . Its performance can also degrade as more models/users are added to a repository \cite{KolovosRMPGCLRV13}.

The summary of the advantages and downsides of the different model persistence solutions presented in this section can be found in Table \ref{table:model_persistence_comparison}. These advantages and downsides give this research some points to consider on how a change-based model persistence format that is compatible with version control systems such as Git and SVN should be designed. These considerations are presented in Section \ref{sec:a_new_change_based_persistence}.

\subsection{Change-based vs. State-based Persistence}
\label{sec:change_based_vs_state_based_persistence}
This section compares the advantages and drawbacks of change-based and state-based persistence. Change-based persistence works by persisting the complete change history of an artefact instead of persisting a snapshot -- the entire state -- of an artefact at a time. The concept of change-based persistence is not new and has been used in persisting changes of software, object-oriented databases, hierarchical documents, and models 
\cite{DBLP:journals/entcs/RobbesL07,DBLP:conf/sde/LippeO92,DBLP:conf/caise/IgnatN05,koegel2010emfstore}. 

Change-based persstence offers two main advantages. First, it records finer-granularity information (e.g. types of changes, the order of the changes, elements that were changed, previous values, etc.) of changes which can improve the accuracy of change detection \cite{DBLP:journals/entcs/RobbesL07,DBLP:conf/sde/LippeO92,DBLP:conf/caise/IgnatN05,mens2002state}. Second, it records changes in an ordered manner which means that changes made to an artefact can be identified sequentially without having to explore and compare all elements of compared versions of an artefact \cite{DBLP:conf/edoc/KoegelHLHD10}. The advantages to detect changes more precisely and much faster can then have positive knock-on effects on supporting (1) developers compare and merge artefacts in collaborative environments \cite{DBLP:conf/sde/LippeO92,DBLP:conf/caise/IgnatN05,koegel2010emfstore}, and (2) incremental management \cite{jouault2010towards,DBLP:conf/ecmdafa/OgunyomiRK15, DBLP:conf/ecmdafa/RathHV12}. Moreover, changed-based persistence contains wealth information which can be exploited for analytics \cite{DBLP:journals/entcs/RobbesL07}.

Nevertheless, change-based persistence also comes with downsides, such as ever-growing artefact files  \cite{DBLP:journals/entcs/RobbesL07,DBLP:conf/edoc/KoegelHLHD10} and increased artefact loading time \cite{mens2002state} which increases storage and computation costs. An artefact that is frequently modified will increase considerably in file size since every change is added to the file. The increased file size (proportional to the number of persisted changes) will, in turn, increase the loading time of the artefact since all changes have to be replayed to reconstruct the artefact's eventual state. 

%These downsides have to be mitigated to enable the practical adoption of change-based persistence. 
%One approach to reducing the file size of change-based models is by removing changes that do not affect the eventual 
%state of the model. For the increased loading time, it can be mitigated by ignoring -- i.e. not replaying -- changes 
%that are cancelled out by later changes or employing change-based and state-based persistence side-by-side so that the
%benefits of state-based persistence on loading time can be obtained. 

Other downsides are change-based persistence requires 
integration with existing tools -- since it is still a non-standard approach -- for its adoption \cite{koegel2010emfstore}, 
and still has limited support for standard, text-based version controls for collaborative development \cite{koegel2010emfstore}. 
These downsides can be addressed by developing a change-based persistence plugin for a specific development environment 
(e.g. Eclipse) and persisting changes in text-based format to support text-based version controls (e.g. Git, SVN).

In summary, state-based persistence has several strong points. First, since it is the default standard persistence approach for most artefacts, it requires minimum effort to integrate with existing tools \cite{koegel2010emfstore}. Second, it is faster in loading artefacts persisted in state-based format since there is no need to replay all changes as in change-based persistence. Also, some of the artefacts support lazy loading. For example, an artefact is not loaded in entirety upfront. Only parts affected by an operation are loaded into memory. This enables faster CRUD (create, read, update, delete) operations \cite{DBLP:conf/models/Espinazo-PaganCM11,daniel2016neoemf}. 

\begin{table*}[h]
  \centering
  \caption{The advantages and downsides between change-based and state-based persistence.}
  \label{table:advantages_drawbacks}
\begin{scriptsize}
    \begin{tabular}
      {|>{\centering\arraybackslash}p{1.1cm}|>{\centering\arraybackslash}p{1.1cm}|>{\centering\arraybackslash}p{5cm}|>{\centering\arraybackslash}p{5cm}|}
      \hline 
      \multicolumn{2}{|c|}{\textbf{Dimensions}}&\textbf{Change-based Approach}&\textbf{State-based Approach}\\
      \hline 
      \multicolumn{2}{|p{2.2cm}|}{\centering Advantages} &
      \begin{minipage}[t]{5cm}
        \begin{itemize}[leftmargin=9pt]
          \setlength\itemsep{2pt}
          \item[+] Faster for detecting changes \cite{DBLP:conf/edoc/KoegelHLHD10}
          \item[+] More accurate, carry semantic information \cite{DBLP:journals/entcs/RobbesL07,DBLP:conf/sde/LippeO92,DBLP:conf/caise/IgnatN05,mens2002state}  
          \item[+] Faster and more accurate for comparison and merging \cite{DBLP:conf/sde/LippeO92,DBLP:conf/caise/IgnatN05,koegel2010emfstore}
          \item[+] Information carried is useful for analytics \cite{DBLP:journals/entcs/RobbesL07}
        \end{itemize}
      \end{minipage}
      & 
      \begin{minipage}[t]{5cm}
        \raggedright
        \begin{itemize}[leftmargin=9pt]
          \setlength\itemsep{2pt}
          \item[+] Faster for loading large artefacts \cite{DBLP:conf/models/Espinazo-PaganCM11,daniel2016neoemf,eclipse2019cdo}
          \item[+] A default standard, no need integration with existing tools \cite{koegel2010emfstore}  
        \end{itemize}
      \end{minipage}
      \\
      \hline
      \multicolumn{2}{|p{2.2cm}|}{\centering Disadvantages} & \begin{minipage}[t]{5cm}
        \raggedright
        \begin{itemize}[leftmargin=9pt]
          \setlength\itemsep{2pt}
          \item[--] Increased record size \cite{DBLP:journals/entcs/RobbesL07,DBLP:conf/edoc/KoegelHLHD10}
          \item[--] Is not efficient for replaying (loading) long records \cite{mens2002state}
          \item[--] Limited supports from standard, text-based version controls (e.g. GitHub) \cite{koegel2010emfstore} 
          \item[--] Not a standard, need integration with existing tools \cite{koegel2010emfstore} 
        \end{itemize}
      \end{minipage}
      & 
      \begin{minipage}[t]{5cm}
        \raggedright
        \begin{itemize}[leftmargin=9pt]
          \setlength\itemsep{2pt}
          \item[--] Slower for saving changes  \cite{mens2002state,daniel2016neoemf,DBLP:conf/models/Espinazo-PaganCM11}
          \item[--] Slower for comparison \cite{DBLP:conf/edoc/KoegelHLHD10}
          \item[--] Less accurate, does not carry semantic information \cite{mens2002state,DBLP:conf/edoc/KoegelHLHD10}  
        \end{itemize}
      \end{minipage}
      \\
      \hline
    \end{tabular} 
  \end{scriptsize}
\end{table*}

Compared to change-based persistence, state-based persistence also has  downsides. First, it is slower than change-based persistence in saving changes \cite{mens2002state}. For an artefact persisted in state-based format and does not support lazy loading, the the artefact has to be persisted in its entirety even thought only a single change has been made. Second, state-based persistence does not have any records of changes of an artefact. Thus, every part of the artefact has to be checked for differences which can be less efficient if the comparison is performed in change-based format \cite{DBLP:conf/edoc/KoegelHLHD10}. Third, since comparison in state-based format requires deriving differences through a diffing process -- not based on actual change records, it can be less accurate than comparison in change-based persistence which is provided with more information to detect changes accurately \cite{mens2002state,DBLP:conf/edoc/KoegelHLHD10}. The summary of the advantages and downsides between change-based and state-based persistence are presented in Table \ref{table:advantages_drawbacks}.

\subsection{Towards a New Change-based Model Persistence}
\label{sec:a_new_change_based_persistence}
The research presented in this thesis aims at improving the performance of model differencing and conflict detection. Based on the literature review of existing model persistence implementations and the advantages and drawbacks of state-based and change-based persistence, this study has identified some considerations that have affected the design of this research and the implementation of the proposed change-based persistence. Points 1 to 4 are addressed in Chapter \ref{ch:change_based_model_persistence}, while point 5 is addressed in Chapter \ref{ch:hybrid_model_persistence}.

\begin{enumerate}
  \item The implementation should exploit change-based persistence to avoid the overhead cost when performing model differencing and conflict detection in state-based persistence.
  \item The implementation should be able to capture and persist relevant changes of models to allow accurate and efficient model differencing and conflict detection as EMF Store does.
  \item Even though EMF Store is the only existing implementation of change-based persistence of models, this research does not consider to use and extend EMF Store since it has some critical downsides that are worth to avoid: (1) This work aims for change-based persistence that can version and share models through different text-oriented version controls (e.g., Git, SVN). In this way, administration activities can be delegated to the version controls, and it can avoid dependency on EMF Store that uses its mechanism to version models. (2) The performance of EMF Store can also degrade as more models/users are added to a repository \cite{KolovosRMPGCLRV13}.
  \item Aiming for compatibility with text-oriented version controls (e.g., Git, SVN) requires a model to be persisted in a text file. This decision can affect the technical implementation, which can be different from EMF Store that wraps changes into packages and persist them in XMI files.
  \item Given that it is not efficient for replaying (loading) long records of change-based persistence, this study also considers incorporating state-based persistence into the implementation to reduce the overhead cost of loading models and also to utilise the features of database backends. EMF Store caches the states of versions to speed up loading models, and most of the state-based persistence with database backends support lazy-loading.
\end{enumerate}

\section{Model Differencing and Conflict Detection}
\label{sec:model_differencing_and_conflict_detection}
The history of model differencing and conflict detection can be traced back to the presence of \textsf{diff} program on Unix or Unix-like platform \cite{hunt1976algorithm}. The program can perform diffing that is comparing text files ``in order to determine how or whether they differ'' \cite{diff}. Diffing basically is about finding the longest common subsequence between two or more sequences which commonly known as the Longest Common Subsequence (LCS) algorithms \cite{bergroth2000lcs}. This problem is equivalent to the Shortest Edit Script (SES) problem that is to find the smallest number of editing (add and delete) in order to make a sequence equal to another sequence \cite{DBLP:journals/algorithmica/Meyers86}. LCS or SES algorithms are commonly implemented by Version Control Systems, such as SVN \cite{svn-diff} and Git \cite{git-diff}, in their \textsf{diff} programs to identify differences between versions of files.   

Applying this diffing approach to some graph-based artefacts, such as XML \cite{w3c-xml} and Ecore models \cite{steinberg2008emf}, is not straightforward since they have different characteristics to text files. For example, XML is a hierarchical document with a tree structure; one node can contains other nodes. The unique feature of XML is that its containment is  unordered whereas in text differencing order is a necessary feature. This has been addressed by Wang et al. \cite{wang2003xdiff} by exploiting key XML  structure characteristics. Identifying differences between Ecore models is even more complex since the models support multiple characteristics of features, such as attribute/reference, literal/object values, single/multiple values, containment/non-containment, etc \cite{steinberg2008emf}. 

There are several existing tools for model differencing. 
EMF Compare \cite{emfcompare2018developer} is a popular tool used to compare and merge EMF models, with generic support for different metamodels. It is an extensible framework so that it can be adapated to the specific needs of certain metamodels. EMF Compare works by performing matching between elements of models being compared and then executing differencing to identify differences between the matched elements. Matching and differencing are discussed in detail in Chapter \ref{ch:model_differencing}. In this study, EMF Compare is used as a baseline for comparative evaluation due to its maturity and ongoing development activity. EMF DiffMerge (EDM) \cite{eclipse2019emfdiffmerge} is similar to EMF Compare except that its abstraction is at a lower level, and it is designed to prevent data loss and enforce model consistency \cite{eclipse2019emfdiffmerge2}. As a consequence, EMF Compare could use the EDM engine when it needs to enforce a particular consistency policy. Also, it supports scoping which means comparison does not have to be at the model level but could also be applied on arbitrarily-defined sets of model elements -- subsets of models -- that can be defined by specific filters \cite{jaxenter2019emfdiffmerge}. EMF Compare is used in this study used for comparative evaluation due to its maturity and ongoing development activity. Other tools, such as SiDiff \cite{Treude2007SiDiff} and DSMDiff \cite{lin2009dsmdiff},  also provide language-agnostic graph-based model comparison, with some room for configuration (e.g., assigning different weights to features of types in the language). Additional expressive power -- at the cost of increased complexity and configuration effort -- is offered by dedicated comparison languages such as the Epsilon Comparison Language, which can be used to compare both homogeneous and heterogeneous models \cite{kolovos2009ecl}. All of these tools work with state-based persistence to identify differences between models.

Our literature review has not revealed any other work that targets comparison of change-based models persisted in text files. Only EMF Store \cite{koegel2010emfstore} identified addresses change-based model conflict detection but it persists models in its own dedicated backend system. Moreover, since it is designed to identify conflicts between changes, it does not give direct, summarised information about which parts of two versions of a model are different -- not for model differencing. Database or dedicated-backend model persistence and version control solutions such as CDO \cite{eclipse2019cdo} and EMF Store provide model comparison capabilities between different versions of the same model but they present integration challenges when users wish to use text-oriented version control systems (e.g. Git, SVN) which are typically file-based and commonly used, and their performance can also degrade as more models/users are added to a repository \cite{KolovosRMPGCLRV13}.

\subsection{The Challenges of Model Comparison}
\label{sec:the_key_challenge_of_incrementality}

The performance of identifying differences between versions of models can become crucial for large evolving models, particularly in the latter phases of the development cycle when many small changes made to models to fine-tune them \cite{selic2003pragmatics}. This challenge has been addressed in incremental model management where changes of models are recorded and used as the basis to perform effective incremental model processing operations. In his work, Egyed \cite{egyed2011automatically} has shown that the property-access recording approach is applicable to query such changes. More recent work has shown that variants of this approach can be used to achieve incrementality in a wide range of model processing operations, including model-to-model transformation \cite{jouault2010towards}, model-to-text transformation \cite{DBLP:conf/ecmdafa/OgunyomiRK15}, model validation, and pattern matching \cite{DBLP:conf/ecmdafa/RathHV12}---as long as changes to models can be precisely identified.  

Nonetheless, this approach works best at identifying differences between serial versions of a model; it does not work straightforwardly to identify differences between parallel -- branched -- versions. In addition, the solutions in incremental model management are coupled to their execution engines, which limits itself to work best only in single-developer environment (discussed more in Section \ref{sec:identifying_changes_in models}). In a collaborative setting, as the sizes and complexity of models grows, it is common to manage a model in multiple parallel versions. Thus, capabilities to identify differences differencing between these parallel versions and to detect conflicts between the differences are also necessary.

Before merging two versions, model differencing and conflict detection need to be executed first in order to identify differences and conflicts between. Nevertheless, performing the model differencing and conflict detection in traditional, state-based way is computationally expensive and memory greedy (discussed more in Section \ref{sec:identifying_changes_in models}). Traditional, state-based model comparison requires every element of the versions being compared to be loaded into memory, matched, and then differenced \cite{emfcompare2018developer}, which is inefficient for large models that only undergo a small number of changes. A novel approach is required that can compare only elements that have been modified -- not all elements -- to speed up model comparison.

\subsection{Identifying Changes in Models}
\label{sec:identifying_changes_in models}
There are two approaches in the literature for identifying changes in models: using notification facilities and model differencing.

\subsubsection{Notifications}
\label{sec:notifications}
In this approach, a model change tracking 
engine needs to hook into the notification facilities 
provided by the modelling tool through which the developer edits the model, 
so that the engine can directly receive notifications as soon as 
changes happen (e.g. class \textsf{Giant} has been deleted, class \textsf{Character} has been renamed to ``Hero''). 
This is an approach taken by the IncQuery incremental pattern matching 
framework \cite{DBLP:conf/ecmdafa/RathHV12} and the ReactiveATL incremental model-to-model 
transformation engine \cite{DBLP:conf/ecmdafa/OgunyomiRK15}. The main advantage of this 
approach is that precise and fine-grained change notifications are provided 
for free by the modelling tool (and thus do not need to be computed by the 
execution engine---which as discussed below can be expensive and inefficient). 
On the downside, this approach is a poor fit for collaborative development 
settings where modelling and automated model processing activities are 
performed by different members of the team.

\subsubsection{Model Differencing}
\label{sec:model_differencing}
  This approach eliminates the coupling between 
modelling tools and model change tracking engines. Instead of depending on 
live notifications, in this approach the developer needs to have access to a copy of the last or other version of the model, so that it can be compared against the current version of 
the model (e.g. using a model-differencing framework such as
EMFCompare\cite{emfcompare2018developer} or EMF DiffMerge \cite{eclipse2019emfdiffmerge}) and the delta can be computed on demand. The main advantage of 
this approach is that it works well in a collaborative development environment 
where typically developers have distinct roles and responsibilities. On the 
downside, model comparison and differencing are computationally expensive and 
memory-greedy as both versions of the model need to be loaded into memory before they can be compared.

In summary, tracking changes of models using notification facilities currently delivers significant performance benefits only in a single-developer environment as the approach is coupled to modelling tools. 
As a result, in collaborative development environments, 
developers need to either forgo the notification approach altogether 
or to work with model differencing, which is computationally expensive and 
memory-greedy.

\subsection{A Novel Solution to Model Change Detection}
\label{sec:a_novel_solution_to_model_change_detection}
A novel solution to model change detection should be able to deliver the advantages of both notification and model differencing approaches while eliminate their drawbacks. To realize such solution, this work comes with an idea that models can be persisted in their complete history of changes, also known as change-based model persistence, as opposed to state-based model persistence that persists models in their snapshots at a time. The change-based model persistence should be text-based. Thus, information about changes of models can be preserved and can be shared across the members of a team through Version Control Systems (e.g., SVN, Git).The proposed change-based model persistence is discussed later in Chapter \ref{ch:change_based_model_persistence}. 

Moreover, change-based model persistence also can be exploited to optimise model comparison to only compares versions of a model on parts that have only been changed since their last shared common version. In other words, not every element of the versions is compared which can lead to a faster model comparison. This solution is discussed later in Chapters \ref{ch:model_differencing} and \ref{ch:conflict_detection}. 

\section{Research Method}
\label{sec:research_method}
In performing this research, this work follows the experiment process proposed by Wohlin et al. \cite{DBLP:books/daglib/0029933/Wohlin}. The experiment process consist of 5 activities: scoping, planning, operation, analysis and interpretation, and presentation and package.

\textbf{Scoping}. In the scoping activity, the hypothesis, goals, and objectives of an experiment have to be defined clearly \cite{DBLP:books/daglib/0029933/Wohlin}. Basili et al. \cite{basili1988tame} provide the following questions (scoping points) in their framework to help determining the scope of an experiment in software engineering: (1) what is studied? (object of study), (2) what is the intention? (purpose), (3) which effect is studied? (quality focus), (4) whose view? (perspective), and where is the study conducted? (context).

\textbf{Planning}. In the planning activity, these components have to be defined in detail: context selection, hypothesis formulation, variables selection, selection of subjects, experiment design selection, instrumentation, and validity evaluation\cite{DBLP:books/daglib/0029933/Wohlin}. The context can be offline vs. online, student vs. professional, toy vs. real problems, specific vs. general. Null and alternative hypotheses have to be stated formally, and the data gathered throughout the experiment should be used -- using appropriate statistical tests -- to reject the null hypothesis if possible. In the variables selection, we determine the independent and dependent variables to be measured. The subjects should be selected carefully that they are representative of the case experimented to generalise the results of the experiment. The experiment should be designed carefully to get the desired results, and suitable standard design types should be selected. Experiment objects, guidelines, and measurement instruments also should be defined to ensure the experiment is executable. Lastly, validity threats should be identified and evaluated.

\textbf{Operation}. The operation activity consists of three steps: preparation, execution, and validation \cite{DBLP:books/daglib/0029933/Wohlin}. 
In the preparation, we select and prepare all the required materials to facilitate the execution of the experiment. The experiment can be executed in several ways, such as in one or multiple occasions, or one or multiyear.  While performing, we have to make sure that the experiment is on the right track, not interrupted, and running correctly. We also have to validate the produced data if they are reasonable and collected orderly.

\textbf{Analysis and interpretation}.
To understand the data gathered, descriptive statistics and visualization can be used. We can also remove unnecessary data and variables to facilitate analysis and interpretation. We can then perform hypothesis testing to reject or accept the experiment's hypotheses. The analysis and interpretation should explain how the data gathered contribute to the rejection or acceptance of the hypotheses. The results might be statistically insignificant, but the lessons might still worth to be learned \cite{DBLP:books/daglib/0029933/Wohlin}. 

\textbf{Presentation and package}. In this activity, experiment results should be documented and published in research papers for the dissemination of the results. The experiment also should be packaged to support other parties to replicate it \cite{DBLP:books/daglib/0029933/Wohlin}. 

\section{Conclusions}
\label{sec:conclusions_2}
This chapter presented a review of literature in the areas of model persistence and differencing. It summarised the advantages and drawbacks of state-based and change-based model persistence, and related work on identifying differences and detecting conflicts between versions of models. This chapter also provided an overview of the research method employed in this research. 

Based on the identified advantages, downsides, and challenges of current approaches on model persistence, differencing, and conflict detection, this chapter has pointed out some design considerations that the change-based persistence proposed in this research should deliver to achieve high-performance model differencing and conflict detection.
