\chapter{Conclusions and Future Work}

\section{Conclusions}

\section{Future Work}


\begin{enumerate}
\item The proposed approach also comes with a number of challenges that this research needs to overcome, such as loading overhead and fast-growing model files. However, the fast-growing model files challenge has not been addressed. 
Persisting models in a change-based format means that model files will keep growing in size during their evolution significantly faster than their state-based counterparts. To address this challenge, (1) we will propose sound change-compression operations (e.g. remove older/unused information) that can be used to reduce the size of a model in a controlled way. (2) We will develop a compact textual format that will minimise the amount of space required to record a change (a textual line-separated format is desirable to maintain compatibility with file-based version control systems). 
\item With appropriate tool support, modellers will be able to ``replay" (part of) the change history of a model (e.g. to understand design decisions made by other developers, for training purposes). In state-based approaches, this can be partly achieved if models are stored in a version-control repository (e.g. Git). However, the granularity would only be at the commit level.
\item By analysing models serialised in the proposed representation, modelling language and tool vendors will be able to develop deeper insights into how modellers actually use these languages/tools in practice and utilise this information to guide the evolution of the language/tool.
\item By attaching additional information to each session (e.g. the id of the developer, references to external documents/URLs), sequences of changes can be traced back to the developer that made them, or to requirements/bug reports that triggered them.
\end{enumerate}