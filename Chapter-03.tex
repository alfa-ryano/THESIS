\chapter{Analysis and Hypothesis}
\label{ch:analysis_and_hypothesis}

This chapter summarises the findings of the literature review and presents the motivation to develop a new change-based persistence format and a novel approach to improve model differencing and conflict detection by exploiting change-based persistence. Based on the findings in Chapter \ref{ch:literature_review}, this chapter presents the hypothesis and research questions addressed in this study. It also presents an overview of the research method used to answer the research questions.

\section{Summary of Findings}
\label{sec:a_new_change_based_persistence}

Performing model differencing and conflict detection in state-based persistence can be expensive in computation time \cite{DBLP:conf/edoc/KoegelHLHD10}. This is because state-based model differencing requires every element of the two versions being compared to be inspected, matched, and diffed to identify their differences \cite{emfcompare2018developer}. Even persisting state-based models using database backends—such as in Teneo \cite{eclipse2017teneo}, CDO \cite{eclipse2019cdo}, Morsa \cite{DBLP:conf/models/Espinazo-PaganCM11}, and NeoEMF \cite{daniel2016neoemf}—can reduce only the overhead cost of loading models, since all elements still need to be checked. Imagine if we have made only small changes on a model, but all of its elements must be examined to identify differences. This approach is not efficient and can cause bottlenecks, especially in collaborative environments where models are often managed in different concurrent versions. Differencing, conflict detection, and merging are common in that context.

As an alternative to state-based persistence, change-based persistence has the potential to deliver high-performance model differencing and conflict detection since the change history of a model is already contained in the model’s change-based representation \cite{DBLP:conf/sde/LippeO92,DBLP:conf/caise/IgnatN05,koegel2010emfstore}. Therefore, identifying changes through model differencing is not required as in state-based persistence. Moreover, model differencing and conflict detection in change-based persistence can also be more accurate than performing them in state-based persistence since the persistent representation also contains detailed information, such as the order of changes, types of changes, and elements affected by changes \cite{DBLP:journals/entcs/RobbesL07,DBLP:conf/sde/LippeO92,DBLP:conf/caise/IgnatN05,mens2002state}.

So far, we have identified EMF Store as the only implementation of change-based model persistence that conforms to the Eclipse Modeling Framework (EMF). However, this research did not use and extend EMF Store for several reasons. First, EMF Store is a full-fledged client-server model repository and versioning system. This means that it requires a certain degree of administration activities (e.g. server configuration, user authentication and authorisation), and it creates a dependency on EMF Store. We favour avoiding such administration activities and dependency and prefer a solution that can version on shared models through different text-oriented version controls (e.g. SVN, Git). Second, it does not scale. There is performance degradation as more models/users are added to a repository and models grow in size \cite{KolovosRMPGCLRV13}. (We have evaluated EMF Store ourselves in Sections \ref{sec:evaluation_6} and \ref{sec:evaluation_discussion}). Third, EMF Store detects conflicts between changes that produce different states when merging. However, it cannot be used directly for model differencing. It is not designed to identify differences between two versions of a model. Fourth, it works only on changes and does not consider eventual states of models in detecting conflicts \cite{DBLP:conf/sfm/BroschKLSWW12}. As a consequence, if an element has been changed concurrently, but the changes produce eventual states that are equal to their original state, EMF Store still treats these changes as though they are in conflict. Last, EMF Store is in maintenance mode. That is, there is no active feature development going on, and its end-of-life might be declared in 2022 \cite{emfstore2019what}.

Based on these considerations, we aimed for a new change-based persistence for EMF-based models. Such an implementation should be able to capture and persist all the changes of models into text-based files, and it should be able to exploit the persisted changes to produce high-performance model differencing and conflict detection.

\section{Hypothesis and Research Questions}
\label{sec:research_questions}
The research in this thesis aims to improve model differencing and conflict detection. Based on the literature review, change-based model persistence has the potential to deliver such performance. To assess whether change-based persistence can improve model differencing and conflict detection, the following hypothesis has been established \textbf{‘a textual change-based model persistence approach can outperform existing model persistence formats in terms of model saving, model differencing, and conflict detection time, with an overhead in terms of model loading time and memory use’}.

In this thesis, the word ‘model’ refers to typed object graphs that conform to three-layer object-oriented meta-modelling architectures such as Eclipse Modeling Framework (EMF) \cite{eclipse2019emf}.

Model differencing is used to identify the differences between versions of a model, such as determining what has been changed from an original version of a model or comparing versions of a model created by different teams working independently. The main goal of conflict detection is to ascertain whether independent updates can be merged, or whether there are conflicts (elements or features that differ in ways that are incompatible) that must first be resolved.

‘Execution time’ as used in the hypothesis is the time required to perform model saving, model differencing, or model conflict detection. We are particularly interested in the benefits and the challenges of using change-based persistence for large models; these are models having more than a million elements as per \cite{daniel2016neoemf,DBLP:conf/models/Espinazo-PaganCM11}. Model loading time is the time required to load a model from its persistent representation into memory. Memory use is the size of the memory occupied during model saving, loading, differencing, and conflict detection.

To assess the validity of the hypothesis, this work aims to answer the following research questions:
\\
\begin{enumerate}
  \item \textbf{How can models be persisted in a change-based format, and how does change-based persistence perform, compared to state-based persistence, in terms of loading and saving models? (RQ1)}
  
  The concept of change-based persistence must be translated into an implementation in a modelling framework context so it can be applied to model persistence, so that its impact on model loading and saving, and later model differencing and model conflict detection can be assessed.
  
  \item \textbf{In a change-based format, how can the differences between models be identified, and how does change-based model differencing perform, in terms of speed and memory footprint, compared to state-based model differencing? (RQ2)}
  
  One of the main motivations for exploring the use of change-based persistence is to speed up model differencing. Because of the nature of change-based persistence, the mechanism to perform change-based model differencing will differ substantially from current state-based model differencing approaches. It is expected that model differencing in change-based persistence will perform faster than model differencing in state-based persistence.
  
  \item \textbf{Following change-based model differencing, how can conflicts be detected between versions of a model, and how does change-based conflict detection perform, in terms of speed and memory, compared to state-based model conflict detection? (RQ3)}
  
  The follow-on effects of change-based persistence on model conflict detection will also be investigated. It is expected that conflict detection of change-based models will be significantly faster than conflict detection of state-based models.
\end{enumerate}

\section{Research Method}
\label{sec:research_method}
In performing this research, this work follows the experimental process proposed by Wohlin et al. \cite{DBLP:books/daglib/0029933/Wohlin}. The experimental process comprises five activities: scoping, planning, operation, analysis and interpretation, and presentation and packaging.

\textbf{Scoping}. In the scoping activity, the hypothesis, goals, and objectives of an experiment must be defined \cite{DBLP:books/daglib/0029933/Wohlin}. Basili et al. \cite{basili1988tame} provide the following questions (scoping points) in their framework to help determine the scope of an experiment in software engineering: (1) what is studied? (object of study), (2) what is the intention? (purpose), (3) which effect is studied? (quality focus), (4) whose view? (perspective), and (5) where is the study conducted? (context).

\textbf{Planning}. In the planning activity, these components must be defined: context selection, hypothesis formulation, selection of variables, selection of subjects, experiment design selection, instrumentation, and validity evaluation\cite{DBLP:books/daglib/0029933/Wohlin}. The context can be offline vs. online, student vs. professional, toy vs. real problems, specific vs. general. Null and alternative hypotheses must be stated formally, and the data gathered throughout the experiment should be used—using appropriate statistical tests—to reject the null hypothesis if possible. The independent and dependent variables to be measured must be determined. The subjects must represent the case being studied so the results of the experiment can be generalised. The experiment must be designed carefully to get the desired results, and suitable standard design types should be selected. Experiment objects, guidelines, and measurement instruments also should be defined to ensure the experiment is executable. Last, validity threats should be identified and evaluated.

\textbf{Operation}. The operation activity comprises three steps: preparation, execution, and validation \cite{DBLP:books/daglib/0029933/Wohlin}.
In the preparation, all the materials needed for the experiment are selected and prepared. The experiment can be executed in several ways, such as once or on multiple occasions, for one year or several years. Execution requires that the experiment is on the right track, not interrupted, and running correctly. Validation means that the data produced must be reasonable and collected orderly.

\textbf{Analysis and interpretation}.
Descriptive statistics and visualisation can be used to understand the data. Unnecessary data and variables can be removed to facilitate analysis and interpretation. Hypothesis testing is used to reject or accept the experiment’s hypothesis. The analysis and interpretation should explain how the data gathered contribute to the rejection or acceptance of the hypothesis. The results might be statistically insignificant, but the lessons might still be worth learning \cite{DBLP:books/daglib/0029933/Wohlin}.

\textbf{Presentation and packaging}. In this activity, the experiment’s results should be documented and published in research papers so they are available to other researchers. The experiment also should be packaged to support other parties who wish to replicate it \cite{DBLP:books/daglib/0029933/Wohlin}.

\section{Conclusions}
\label{sec:conclusions_2b}
Chapter 2 presented the advantages, downsides, and challenges of current approaches to model persistence, differencing, and conflict detection in the scientific literature. In this chapter, we have pointed out design considerations that any proposed solution should deliver to achieve high-performance model differencing and conflict detection. From there, we established the hypothesis and research questions of this study. Finally, we presented an overview of the research method used in this research.

