\chapter{Analysis and Hyphothesis}
\label{ch:analysis_and_hypothesis}

This chapter reflects on the findings of the literature review and presents the motivation for a new change-based persistence format and a novel approach to improve the performance of model differencing and conflict detection by exploiting the change-based persistence. Based on the findings in Chapter \ref{ch:literature_review}, this chapter presents the hypothesis and research questions addressed in this study. It also presents an overview of the employed research method in answering the research questions. 

\section{Towards a New Change-based Model Persistence, Differencing, and Conflict Detection}
\label{sec:a_new_change_based_persistence}
The research presented in this thesis aims at improving the performance of model differencing and conflict detection. 
%To achieve the aim, based on the literature review of existing model persistence implementations, the advantages and drawbacks of state-based and change-based persistence, and current methods for identifying changes of models in Sections \ref{sec:model_persistence} \ref{sec:sec:model_differencing_and_conflict_detection}
Based on the literature review, change-based model persistence has the potential to deliver such performance. However, its drawbacks should be mitigated, and a novel approach also should be defined to use change-based persistence to improve the performance of model differencing and conflict detection. This study has identified some considerations to deliver such solution. Points 1 to 4 are addressed in Chapter \ref{ch:change_based_model_persistence}, while point 5 is addressed in Chapters \ref{ch:optimised_loading} and \ref{ch:hybrid_model_persistence}. Point 6 and 7 are discussed later in Chapters \ref{ch:model_differencing} and \ref{ch:conflict_detection}. 

\begin{enumerate}
  \item The implementation should exploit change-based persistence to avoid the overhead cost when performing model differencing and conflict detection in state-based persistence.
  \item The implementation should be able to capture and persist relevant changes of models to allow accurate and efficient model differencing and conflict detection as EMF Store does.
  \item Even though EMF Store is the only existing implementation of change-based persistence of models, this research does not consider to use and extend EMF Store since it has some critical downsides that are worth to avoid: (1) This work aims for change-based persistence that can version and share models through different text-oriented version controls (e.g., Git, SVN). In this way, administration activities can be delegated to the version controls, and it can avoid dependency on EMF Store that uses its mechanism to version models. (2) The performance of EMF Store can also degrade as more models/users are added to a repository \cite{KolovosRMPGCLRV13}.
  \item Aiming for compatibility with text-oriented version controls (e.g., Git, SVN) requires a model to be persisted in a text file. This decision can affect the technical implementation, which is can be different from the EMF Store's implementation that wraps changes into packages and persist them in XMI files.
  \item Given that it is not efficient for replaying (loading) long records of change-based persistence, the loading can be optimised by not replaying superseded changes. This study also considers incorporating state-based persistence into the implementation to reduce the overhead cost of loading models and also to utilise the features of database backends. EMF Store caches the states of versions to speed up loading models, and most of the state-based persistence with database backends support lazy-loading.
  \item The solution should be able to retrieve information, from change-based persistence representation, of which parts of a model that have been modified and what kind of changes that have been applied.
  \item The solution should be able to make use such information to identify the differences between the two versions of a model being compared and the conflicts between the changes applied to them.
\end{enumerate}


\section{Research Questions}
\label{sec:research_questions}
In order to proof that change-based persistence can be used to improve the performance of model differencing and conflict detection, a hypothesis has been established. The hypothesis of this research is that \textbf{``a textual change-based model persistence approach can outperform existing model persistence formats in terms model saving, model differencing and conflict detection time, with an overhead in terms of model loading time and memory use''}. 

In this thesis, the word 'model' refers to typed object graphs that conform to 3-layer object-oriented metamodelling architectures such as Eclipse Modelling Framework (EMF) \cite{eclipse2019emf}. 

Model differencing is used to identify the differences between versions of a model. For example, determining what has been changed from an original version of a model, or comparing versions of a model created by different teams working independently. The main goal of conflict detection is to ascertain whether independent updates can be merged, or whether there are conflicts that needs to be resolved first: elements or features that differ in ways that are incompatible. 

The execution time referred in the hypothesis is the time required perform model saving, model differencing, or model conflict detection. We are parcicularly interested in the benefits and the challenges of using change-based persistence for large models; these are models having more than a million elements as per \cite{daniel2016neoemf,DBLP:conf/models/Espinazo-PaganCM11}. Model loading time is the amount of time required to load a model from its persistence representation into memory. Memory use is the size of the memory occupied at the time of model saving, loading, differencing, and conflict detection.

To assess the validity of the hypothesis, this work aims to answer the following research questions: 
\begin{enumerate} 
  \item \textbf{How can models be persisted in a change-based format, and how does change-based persistence perform, compared to state-based persistence on loading and saving models? (RQ1)} 
  
  The concept of change-based persistence has to be translated into an implementation in a modelling framework context so that it can be applied for model persistence, and therefore its impact on model loading and saving, and later model differencing, and model conflict detection can be assessed.
  
  \item \textbf{In a change-based format, how can differences between models be identified, and how does change-based model differencing perform, in terms of speed and memory footprint, compared to state-based model differencing? (RQ2)} 
  
  One of the main motivations for exploring of using change-based persistence in this work is to speed up model differencing. Due to the nature of change-based persistence, the mechanism to perform change-based model differencing will differ substantially from the current state-based model differencing approaches. It is expected that model differencing in change-based persistence will perform faster than model differencing in state-based persistence.        
  
  \item \textbf{Following change-based model differencing, how can conflicts be detected between versions of a model, and  how does change-based conflict detection perform, in terms of speed and memory, compared to state-based model conflict detection? (RQ3)} 
  
  The knock-on effect of change-based persistence on model conflict detection will also be investigated. It is expected that conflict detection of change-based models will be significantly faster than conflict detection of state-based models.
\end{enumerate}

\section{Research Method}
\label{sec:research_method}
In performing this research, this work follows the experiment process proposed by Wohlin et al. \cite{DBLP:books/daglib/0029933/Wohlin}. The experiment process consist of 5 activities: scoping, planning, operation, analysis and interpretation, and presentation and package.

\textbf{Scoping}. In the scoping activity, the hypothesis, goals, and objectives of an experiment have to be defined clearly \cite{DBLP:books/daglib/0029933/Wohlin}. Basili et al. \cite{basili1988tame} provide the following questions (scoping points) in their framework to help determining the scope of an experiment in software engineering: (1) what is studied? (object of study), (2) what is the intention? (purpose), (3) which effect is studied? (quality focus), (4) whose view? (perspective), and where is the study conducted? (context).

\textbf{Planning}. In the planning activity, these components have to be defined in detail: context selection, hypothesis formulation, variables selection, selection of subjects, experiment design selection, instrumentation, and validity evaluation\cite{DBLP:books/daglib/0029933/Wohlin}. The context can be offline vs. online, student vs. professional, toy vs. real problems, specific vs. general. Null and alternative hypotheses have to be stated formally, and the data gathered throughout the experiment should be used -- using appropriate statistical tests -- to reject the null hypothesis if possible. In the variables selection, the independent and dependent variables to be measured are determined. The subjects should be selected carefully that they are representative of the case experimented to generalise the results of the experiment. The experiment should be designed carefully to get the desired results, and suitable standard design types should be selected. Experiment objects, guidelines, and measurement instruments also should be defined to ensure the experiment is executable. Lastly, validity threats should be identified and evaluated.

\textbf{Operation}. The operation activity consists of three steps: preparation, execution, and validation \cite{DBLP:books/daglib/0029933/Wohlin}. 
In the preparation, all the required materials to facilitate the execution of the experiment are selected and prepared. The experiment can be executed in several ways, such as in one or multiple occasions, or one or multiyear.  While performing, the experiment has to be made sure if it is on the right track, not interrupted, and running correctly. The produced data also have to be validated if they are reasonable and collected orderly.

\textbf{Analysis and interpretation}.
To understand the data gathered, descriptive statistics and visualization can be used. Unnecessary data and variables also can be removed to facilitate analysis and interpretation. Hypothesis testing can be performed to reject or accept the experiment's hypotheses. The analysis and interpretation should explain how the data gathered contribute to the rejection or acceptance of the hypotheses. The results might be statistically insignificant, but the lessons might still worth to be learned \cite{DBLP:books/daglib/0029933/Wohlin}. 

\textbf{Presentation and package}. In this activity, experiment results should be documented and published in research papers for the dissemination of the results. The experiment also should be packaged to support other parties to replicate it \cite{DBLP:books/daglib/0029933/Wohlin}. 

\section{Conclusions}
\label{sec:conclusions_2b}
In this chapter, reflecting on the identified advantages, downsides, and challenges of current approaches on model persistence, differencing, and conflict detection in the literature review, we have pointed out some design considerations that the solution proposed in this research should deliver to achieve high-performance model differencing and conflict detection. From there, we established the hypothesis and research questions of this study. Finally, we presented the overview of the research method employed in this research. 