\chapter{Introduction}
\label{ch:introduction}
This Chapter briefly presents the background of this work as well as the research questions that will be 
addressed in this project. Several research objectives are then defined to answer the research questions. 
Lastly, research outputs and scoping are also presented. 

\section{Background}
\label{sec:background}
Most of the models in the context of Model-Driven Engineering are persisted in state-based formats. 
In such approaches, model files contain snapshots of the models' contents, and activities like version control 
and change detection are left to external systems such as file-based version-control systems and model differencing 
facilities. Activities such as change-detection (identifying parts that have changed in a model compared 
to a previous version) and model comparison (finding differences between models) are computationally consuming
for state-based models \cite{Kolovos:2009:DMM:1564596.1564641}. Thus, a new approach is needed to make the 
computation more efficient.

As an alternative to state-based persistence, this work proposes that a model can also be persisted in a change-based format, 
which persists the full sequence of \emph{changes} made to the model instead. 
The concept of change-based persistence is not new and has been used in persisting changes to software, 
object-oriented databases, and hierarchical documents 
\cite{DBLP:journals/entcs/RobbesL07,DBLP:conf/sde/LippeO92,DBLP:conf/caise/IgnatN05}. 
The change-based approach can improve detecting differences more precisely at the semantic 
level -- that is by providing finer-granularity information (e.g. types of changes, the order of the changes, 
elements that were changed, previous values, etc.) -- and therefore provide support to resolve them \cite{mens2002state}. 
The ordered nature of change-based persistence means that changes made to a model can be identified sequentially without 
having to explore and compare all elements of the model and its previous version. Based on these arguments, 
this work explores the advantages and shortcomings of change-based persistence as an alternative approach to 
state-based persistence for models conforming to 3-layer metamodelling architectures such as EMF and MOF. 
Persisting models in a change-based format can bring a number of envisioned benefits over state-based persistence, 
such as the ability to detect changes much faster and more precisely, which can then have positive 
knock-on effects on supporting (1) developers compare and merge models in collaborative modelling environments, 
and (2) incremental model management ( e.g. incremental query \cite{DBLP:conf/ecmdafa/RathHV12} and 
model-to-text transformation \cite{DBLP:conf/ecmdafa/OgunyomiRK15}). 

Nevertheless, change-based persistence also comes with downsides, such as ever-growing model files 
\cite{DBLP:journals/entcs/RobbesL07,DBLP:conf/edoc/KoegelHLHD10} and increased model loading time \cite{mens2002state}
which increase storage and computation costs. A model that is frequently modified will increase considerably in file size 
since every change is added to the file. The increased file size (proportional to the number of persisted changes) will, 
in turn, increase the loading time of the model since all changes have to be replayed to reconstruct the model's 
eventual state. These downsides have to be mitigated to enable the practical adoption of change-based persistence. 
One approach to reducing the file size of change-based models is by removing changes that do not affect the eventual 
state of the model. For the increased loading time, it can be mitigated by ignoring -- i.e. not replaying -- changes 
that are cancelled out by later changes or employing change-based and state-based persistence side-by-side so that the
benefits of state-based persistence on loading time can be obtained. Other downsides are change-based persistence requires 
integration with existing tools -- since it is still a non-standard approach -- for its adoption \cite{koegel2010emfstore}, 
and still has limited support for standard, text-based version controls for collaborative development \cite{koegel2010emfstore}. 
These downsides can be addressed by developing a change-based persistence plugin for a specific development environment 
(e.g. Eclipse) and persisting changes in text-based format to support text-based version controls (e.g. Git, SVN).

\section{Research Questions}
\label{sec:research_questions}
The hypothesis of this work is that \textbf{``Change-based persistence reduces the execution time of model change-detection, model comparison, and model merging for large models compared to their execution time in state-based persistence, with acceptable trade-offs on loading and persisting time, memory footprint, and storage space consumption''}. The execution time is the time required to complete the processes (e.g. change-detection, model comparison, model merging, or persisting changes). Model change-detection is identifying changed elements of a model compared to its previous version/ancestor while the model comparison is finding the differences between two models that come from the same ancestor. Model merging is reconciling two models that come from the same ancestor and combining them to produce a new model. Using the term ``large models'' we refer to models with more than 1M elements, consistently with \cite{daniel2016neoemf,DBLP:conf/models/Espinazo-PaganCM11}. Model load time is the amount of time required to load a model into memory. Persisting changes is saving changes made to a model into a persistent representation (e.g. a file). Memory footprints are the sizes of memory used to execute the processes. Disk space consumption is the amount of storage consumed by the persistence. The term ``acceptable'' means the cost for memory extension is significantly lower than the cost for extending processor to reach the same execution time, and the loading and persisting time should be faster or slightly slower than the state-based approach.\\

To assess the validity of the hypothesis, this work aims to answer the following research questions: 
\begin{enumerate} 
\item \textbf{How to persist models in a change-based format? How does it perform compared to state-based persistence on saving changes?} 

The concept of change-based persistence has to be translated into an implementation in a modelling framework context so that it can be applied for model persistence, and therefore its impact on model change-detection, model comparison, and model merging can be assessed.

\item \textbf{How to detect changes in change-based models -- comparing them to their ancestors/previous versions? To what extent does change-detection in change-based models perform compared to change-detection in state-based models?} 

The purpose of using change-based persistence in this work is to improve change-detection. The change-based persistence will have change-detection time that is smaller than the change-detection time of state-based persistence.        

\item \textbf{How to compare change-based models that come from the same ancestor? How does the comparison of change-based models perform compared to the state-based model comparison?} 

The knock-on effect of faster change-detection on model comparison will also be investigated. Due to the nature of change-based models, the mechanism to perform change-based model comparison will differ substantially from the current state-based model comparison. It is expected that comparison of change-based models will be significantly faster than the comparison of state-based models.

\item \textbf{How to merge different change-based models that come from the same ancestor? How does the merging perform compared to model merging in state-based persistence?}

Another knock-on effect of faster change-detection of change-based persistence is faster model merging. Similar to the change-based model comparison, the mechanism to merge change-based models will differ substantially from merging state-based models. It is expected that the change-based model merging will be much faster than state-based model merging.   

\end{enumerate}

\section{Research Objectives}
\label{sec:research_objectives}
This research aims to meet the following research objectives to answer the research questions.
\begin{enumerate}
\item Develop an implementation of change-based persistence so it can be applied to persist models in change-based format, and evaluate the correctness of change-based models that it produces and its performance on saving changes against state-based persistence. 
\item Develop a solution to detect changes in change-based models, and compare its execution time and memory footprint against change-detection in state-based models.
\item Develop a solution to compare change-based models, and compare its execution time and memory footprint against model comparison in state-based models.
\item Develop a solution to merge different change-based models, and compare its execution time and memory footprint against model-merging in state-based models. 
\end{enumerate}

\section{Research Outputs}
\label{sec:research_outputs}
By the end of this research, these following outputs will have been produced:
\begin{enumerate}
\item Prototypes for change-based persistence. 
\item Solutions -- including their implementation and evaluation -- for file size and loading time reduction, change-detection (finding parts that already changed of a model compared to its previous version/ancestor), model comparison (finding differences between models that come from the same ancestor), and model merging of change-based persistence.
\item Publications and a thesis documenting the outcomes of this research.
\end{enumerate}


\section{Research Scope}
\label{sec:research_scope}
The scope of this research will be restricted to models conforming to 3-level metamodelling architectures. The Eclipse Modelling Framework will be used, as a representative example of such architectures, for the implementation of all algorithms and prototypes.

\chapter{Thesis Structure}
\label{sec:Thesis Structure}
This section provides the description of the planned thesis structure and its progress. Each subsection corresponds to a chapter in the thesis report and describes the purpose, contents, and progress of the chapter. Some of the contents have been discussed in this research's Progress Report \cite{yohannis2017progress}, Thesis Outline \cite{yohannis2018outline}, and written papers \cite{DBLP:conf/models/YohannisKP17,yohannis2018towards,DBLP:conf/models/YohannisRPK18}, and they will be transferred to the thesis report with proper adjustments. The planned time frame for each chapter is presented in Chapter \ref{ch:plan}.

\section{Chapter 1: Introduction}
\label{sec:chapter_1_introduction_plan}
This chapter is intended to present the motivation and purpose of this research. It will comprise the (1) background of the research as well as the (2) research hypothesis, (3) research questions, (4) research objectives, (5) research outputs, and (6) research scope. All of these contents have been defined in the Progress Report \cite{yohannis2017progress} and  this Thesis Audit, and they will be transferred to the Thesis Report with proper adjustments. 

\section{Chapter 2: Literature Review}
\label{sec:chapter_2_literature_review_plan}
This chapter is dedicated to summarise work related to change-based persistence and comparison, critically assesses the advantages and disadvantages of current approaches, and seek the opportunities to contribute novel knowledge to the field. The chapter will comprise (1) change-based approaches in software engineering, (2) change-based model persistence, (3) state-based model persistence, (4) text-based comparison and merging, (5) state-based model comparison and merging, and (6) change-based model comparison and merging. Most of these topics have been discussed briefly in the Progress Report \cite{yohannis2017progress}. %However, text-based comparison, state-based model comparison, and change-based model comparison and merging need more literature review and deeper analysis to support defining a new approach for change-based model comparison and merging.

\section{Chapter 3: Change-based Model Persistence}
\label{sec:chapter_3_Change-based_model_ersistence_plan}
This chapter is planned to present the concept of change-based model persistence and its core implementation. The chapter will comprise the concept of change-based model persistence, and the design of the core implementation. The implementation is evaluated by comparing the eventual model produced by replaying a change-based representation to the same model but loaded from a state-based persistence (e.g. XMI). These contents have been published in a workshop \cite{DBLP:conf/models/YohannisKP17}.


\section{Chapter 4: Optimised Loading of Change-based Models}
\label{sec:chapter_4_optimised_loading_change_based_model_persistence}

Change-based persistence comes with a downside of ever-growing file sizes \cite{DBLP:journals/entcs/RobbesL07,DBLP:conf/edoc/KoegelHLHD10} which causes increased loading time \cite{mens2002state}. Reducing the loading time is essential to facilitate the practical adoption of change-based persistence. One way to reduce the increased loading time is by ignoring -- not replaying -- changes that are cancelled out by subsequent changes. 

To evaluate the efficiency of the proposed approach, the optimised loading is compared to a naive loading of a change-based representation and loading the same model from a state-based representation. They are compared on time required to load models and the memory footprint after loading models. Evaluation is also performed on time required for persisting changes between change-based and state-based persistence to show the benefit of change-based persistence on saving changes. 

This chapter is intended to present the optimisation approach along with its implementation and evaluation. The contents of this chapter will be largely based on an accepted conference paper \cite{yohannis2018towards}. 


\section{Chapter 5: Hybrid Model Persistence}
\label{sec:chapter_5_hybrid_model_persistence}
While optimised loading is faster than the naive loading, the benefits are moderate and that the optimised loading is still slower than loading from a state-based representation \cite{yohannis2018towards}. This finding has motivated the design and development of a hybrid persistence approach, that is augmenting a change-based representation with a state-based representation. 

The hybrid model persistence approach is evaluated by comparing it to state-based persistence (e.g. XMI, NeoEMF \cite{daniel2016neoemf}) on time, memory footprint, and storage space required for loading models and persisting changes. The evaluation is also performed on time required for detecting changes between hybrid and state-based persistence to show the benefit of change-based persistence over state-based model comparison. 

This chapter is dedicated to present the hybrid model persistence approach together with its implementation and evaluation. The contents of this chapter will be largely based on a workshop paper \cite{DBLP:conf/models/YohannisRPK18} that is currently under review. 

\section{Chapter 6: Change-based Model Comparison}
\label{sec:chapter_6_change_based_model_comparison}
This chapter will present a change-based model comparison algorithm with its implementation and evaluation. Change-based persistence is expected to speed-up model comparison because the information required to identify changes is already contained in the models' representation. 

The proposed algorithm will be evaluated by comparing it to state-based model comparison (e.g. EMF Compare \cite{eclipse2017compare}) on the time and memory footprint required to find all differences between models. This research will derive synthetic models from branches of real-world software projects as datasets in the evaluation. The alternative strategy is to perform an experiment on modellers. They will be asked to develop models in parallel. The produced models will be upscaled and used in the evaluation. 

The findings of this chapter will be published together with the findings of Chapter 7.


\section{Chapter 7: Change-based Model Merging}
\label{sec:chapter_7_change_based_model_Merging}
This chapter is planned to present change-based model merging. After identifying the differences between two versions of a change-based model, there is a need to reconcile all the differences into a new version of the model. Thus, conflicts between the two versions have to be identified and resolved -- using predefined rules or based on users' decisions, and by then merging steps can be determined to merge the two versions. 

Similar to change-based model comparison in Section \ref{sec:chapter_6_change_based_model_comparison}, the proposed merging approach will also be evaluated by comparing it to the merging of state-based persistence (e.g. EMF Compare) on the affected time and memory footprint. The evaluation will also use the same datasets in Section \ref{sec:chapter_6_change_based_model_comparison}.

The findings of this chapter will be published together with the findings of Chapter 6.