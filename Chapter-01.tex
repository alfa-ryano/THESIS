\chapter{Introduction}
\label{ch:introduction}
This Chapter briefly presents the background of to the research presented in this thesis as well as the aim of this research. Several research objectives are then defined to achieve the research aim. Research outputs and scoping are also presented. Lastly, this chapter presents the structure of this thesis as well as the papers that have been published in this research.

\section{Background}
\label{sec:background}
In large-scale computer systems and software development, model-driven engineering is an approach that focuses on the development and management of models -- usually expressed in diagrammatic or textual notations. Most of these models are persisted in state-based formats. In a state-based format, model files contain snapshots of the models' contents, and activities like version control and change detection are left to external systems such as file-based version-control systems and model differencing facilities. Activities such as model differencing (identifying parts of two versions of a model that are different) and conflict detection (finding conflicting changes between two versions of a model) are computationally expensive for state-based models \cite{Kolovos:2009:DMM:1564596.1564641}. The research presented in this thesis is motivated by the need to find a more efficient approach to model differencing and conflict detection.

As an alternative to state-based persistence, this work proposes that a model can be persisted in a change-based format, 
which persists the full sequence of \emph{changes} made to the model. 
The concept of change-based persistence has been previously used in persisting changes to software, 
object-oriented databases, and hierarchical documents 
\cite{DBLP:journals/entcs/RobbesL07,DBLP:conf/sde/LippeO92,DBLP:conf/caise/IgnatN05}. Change based approaches facilitate detection of differences between versions, and makes better semantic identification of differences -- that is by providing finer-granularity information (e.g. types of changes, the order of the changes, elements that were changed, previous values, etc.).  Better and more granular identification of differences can enable better support for resolution of conflicts e.g. where versions of a model have been modified in different ways \cite{mens2002state}. 
The ordered nature of change-based persistence means that changes made to a model can be identified sequentially without having to explore and compare all elements of the model and its previous version. The ability to detect changes faster and precisely can then have positive knock-on effects on supporting (1) model differencing, conflict detection, and merging in collaborative modelling environments, and (2) incremental model management (e.g., incremental query \cite{DBLP:conf/ecmdafa/RathHV12} and model-to-text transformation \cite{DBLP:conf/ecmdafa/OgunyomiRK15}). Based on these arguments, 
this work explores the advantages and shortcomings of change-based persistence as an alternative approach to state-based persistence for models conforming to 3-layer metamodelling architectures such as Eclipse Modelling Framework (EMF) \cite{eclipse2019emf} and Meta-Object Facility (MOF) \cite{omg2018mof}.

Nevertheless, change-based persistence also comes with downsides, such as ever-growing model files \cite{DBLP:journals/entcs/RobbesL07,DBLP:conf/edoc/KoegelHLHD10} and increased model loading time \cite{mens2002state}
which increase storage and computation costs. Every time a model is modified, the file that records its list of changes increases in size. The increased file size (proportional to the number of persisted changes),  in turn, increase the loading time of the model since all changes have to be replayed to reconstruct the model's eventual state. These downsides need to be mitigated to enable the practical adoption of change-based persistence. Another downside is that change-based persistence requires integration with existing tools for its adoption \cite{koegel2010emfstore} since it is still a non-standard approach. This downside can be addressed by developing a change-based persistence plugin for a specific development environment (e.g. Eclipse).
%It also still has limited support for standard, text-based version controls for collaborative development \cite{koegel2010emfstore} and persisting changes in text-based format to support text-based version controls (e.g. Git, SVN).

\section{Research Aim}
\label{sec:research_aim}
The aim of this work is to assess the advantages and shortcomings of a novel change-based model persistence approach against existing textual and database-backed state-based and database-backed changed-based model persistence formats. This work is concerned with models conforming to metamodels expressed in object-oriented metamodelling languages such as Ecore and MOF. The advantages and shortcomings considered in this work are in terms of computational cost and memory usage for 1) model loading, 2) model saving, 3) model differencing and 4) conflict detection.

\section{Research Objectives}
\label{sec:research_objectives}
This research has defined the following research objectives to achieve the research aim.
\begin{enumerate}
\item To identify and study existing model persistence approaches in the context of the EMF metamodelling architecture.
\item To identify and study change-based artefact persistence approaches beyond EMF.
\item To design a generic change-based model persistence format for models conforming to arbitrary EMF (Ecore) metamodels, and to implement algorithms for saving and loading models in the said format.
\item To implement algorithms for differencing and conflict detection between pairs of versions of change-based models.
\item To assess the performance and memory use of loading, saving, differencing and conflict detection of change-based models against established model persistence approaches within EMF.
%\item Research and design an implementation of change-based persistence so it can be applied to persist models in change-based format, and evaluate the correctness of change-based models that it produces and its performance on saving changes against state-based persistence. 
%\item Develop a solution to perform change-based model differencing, and compare its execution time and memory footprint against state-based model differencing.
%\item Develop a solution to perform change-based model conflict detection, and compare its execution time and memory footprint against state-based model conflict detection.
\end{enumerate}

\section{Research Outputs}
\label{sec:research_outputs}
By the end of this research, these following outputs have been produced:
\begin{enumerate}
\item Prototypes of change-based persistence, change-based model differencing, and change-based conflict detection. 
\item Designs and evaluation results of novel approaches for loading time reduction and saving, model differencing, and model conflict detection of change-based models.
\item Publications \cite{DBLP:conf/models/YohannisKP17,yohannis2018towards,DBLP:conf/models/YohannisRPK18,yohannis2019efficient} and a thesis documenting the solutions proposed in this research.
\end{enumerate}

\section{Research Scope}
\label{sec:research_scope}
The scope of this research is as follows:
\begin{enumerate}
\item This work is restricted to models conforming to 3-level metamodelling architectures. The Eclipse Modelling Framework is used, as a representative example of such architectures, for the implementation of all solutions and prototypes.
\item This work only covers change-based model persistence, differencing, and conflict detection. Change-based model merging is beyond the scope of the research presented in this thesis.
\item Although it is mentioned several times in this report, the use of change-based persistence to support incremental model management is not part of this work. 
\end{enumerate}

\section{Thesis Structure}
\label{sec:Thesis Structure}
This section provides the overview of the remaining chapters of the thesis.

%\subsection{Chapter 1: Introduction}
%\label{sec:chapter_1_introduction_plan}
%This chapter is intended to present the motivation and purpose of this research. It comprises the background of the research as well as the research hypothesis, research questions, research objectives, research outputs, research scope, thesis structure, and published papers. 

\subsection{Chapter 2: Literature Review}
\label{sec:chapter_2_literature_review_plan}
This chapter summarises work related to change-based persistence and comparison, critically assesses the advantages and disadvantages of current approaches, and seeks opportunities to contribute novel knowledge to the field. The chapter comprises brief discussion on methods to identify changes in models, the benefits of change-based solutions to model management, the advantages and drawbacks of state and change-based persistence in the context of software engineering, the state of the art in model persistence, the state of the art in model comparison, state-based model differencing, state-based conflict detection, change-based conflict detection, and the research method applied in this work.

\subsection{Chapter 3: Analysis and Hypothesis}
\label{sec:ch:analysis_and_hypothesis}
This chapter summarises work related to change-based persistence and comparison, critically assesses the advantages and disadvantages of current approaches, and seeks opportunities to contribute novel knowledge to the field. The chapter comprises brief discussion on methods to identify changes in models, the benefits of change-based solutions to model management, the advantages and drawbacks of state and change-based persistence in the context of software engineering, the state of the art in model persistence, the state of the art in model comparison, state-based model differencing, state-based conflict detection, change-based conflict detection, and the research method applied in this work.

\subsection{Chapter 4: Designing Change-based Persistence for Models}
\label{sec:chapter_3_Change-based_model_ersistence_plan}
This chapter presents the concept of change-based model persistence and its prototype implementation. These contents have been published in a workshop \cite{DBLP:conf/models/YohannisKP17}.

\subsection{Chapter 5: Optimised Loading of Change-based Models}
\label{sec:chapter_4_optimised_loading_change_based_model_persistence}

Change-based persistence comes with a downside of ever-growing file sizes \cite{DBLP:journals/entcs/RobbesL07,DBLP:conf/edoc/KoegelHLHD10} which causes increased loading time \cite{mens2002state}. Reducing the loading time is essential to facilitate the practical adoption of change-based persistence. One way to reduce the increased loading time is by ignoring -- not replaying -- changes that are cancelled out by subsequent changes. 

To evaluate the efficiency of the proposed approach, an optimised loading algorithm that ignores superseded change events is compared to a naive loading of a change-based representation and loading the same model from a state-based representation. They are compared on time required to load models and their memory footprints. Evaluation is also performed on time required for persisting changes between change-based and state-based persistence. The contents of this chapter is largely based on a published conference paper \cite{yohannis2018towards}. 

\subsection{Chapter 6: Hybrid Model Persistence}
\label{sec:chapter_5_hybrid_model_persistence}
While optimised loading is faster than naive loading, the benefits are moderate and optimised loading is still slower than loading from a state-based representation \cite{DBLP:conf/models/YohannisRPK18}. This finding has motivated the design and development of a hybrid persistence approach, that augments a change-based representation with a fully derived, state-based representation. 

The hybrid model persistence approach is evaluated by comparing it to state-based persistence (e.g. XMI, NeoEMF \cite{daniel2016neoemf}) on time, memory footprint, and storage space required for loading models and persisting changes. The evaluation is also performed on time required for detecting changes between hybrid and state-based persistence. The contents of this chapter are largely based on a conference paper \cite{DBLP:conf/models/YohannisRPK18}.

\subsection{Chapter 7: Efficient Model Differencing of Change-based Models}
\label{sec:chapter_6_model_differencing}
This chapter presents a change-based model differencing with its implementation and evaluation. Change-based persistence is expected to speed-up model differencing because the information required to identify changes is already contained in the models' persistence. 

The proposed model differencing is evaluated by comparing it to state-based model differencing on the time and memory footprint required to find all differences between two versions of a model. The contents of this chapter are largely based on a workshop paper \cite{yohannis2019efficient}.

\subsection{Chapter 8: Efficient Conflict Detection of Change-based Models}
\label{sec:chapter_7_conflict_detection}
This chapter presents change-based model conflict detection. After identifying the differences between two versions of a change-based model, this work also aims to detect conflicts between two versions of a model. Model conflict detection is a crucial step that precedes model merging.

Similar to change-based model comparison, the proposed conflict detection is also evaluated by comparing it to the conflict detection of existing change and state-based persistence on the affected time and memory footprint.

\subsection{Chapter 9: Conclusions and Future Work}
\label{sec:chapter_8_conclusions_and_future_work}
This chapter summarises the work that has been carried out and the evaluation results obtained in order to answer the research requestions and hypothesis proposed in Section \ref{sec:research_questions}. It also states limitations and threats to the validity of the findings of this thesis and also presents future work that address these.

\section{Publications}
\label{sec:publications}
The research in various parts of the thesis has been published in the following 
papers: 
\begin{enumerate}
  \item A. Yohannis, D. S. Kolovos, and F. Polack, ``Turning models inside out,'' in Proceedings of MODELS 2017 Satellite Events co-located with ACM/IEEE 20th International Conference on Model Driven Engineering Languages and Systems (MODELS 2017), Austin, TX, USA, September, 17, 2017., 2017, pp. 430–434. [Online]. Available: \url{http://ceur-ws.org/Vol-2019/flexmde_8.pdf} (Chapter 3).
  
  \item  A. Yohannis, H. H. Rodriguez, F. Polack, and D. S. Kolovos, ``Towards efficient loading of change-based models,'' in Modelling Foundations and Applications - 14th European Conference, ECMFA 2018, Held as Part of STAF 2018, Toulouse,  France, June 26-28, 2018, Proceedings, 2018, pp. 235–250. [Online]. Available: \url{https://doi.org/10.1007/978-3-319-92997-2_15} (Chapter 4).
  
  \item  A. Yohannis, H. H. Rodriguez, F. Polack, and D. S. Kolovos, ``Towards hybrid model persistence,'' in Proceedings of MODELS 2018 Workshops co-located with ACM/IEEE 21st International Conference on Model Driven Engineering Languages and Systems (MODELS 2018), Copenhagen, 
  Denmark, October, 14, 2018., 2018, pp. 594–603. [Online]. Available:  \url{http://ceur-ws.org/Vol-2245/me_paper_3.pdf} (Chapter 5).
  
  \item  A. Yohannis, H. H. Rodriguez, F. Polack, and D. Kolovos,``Towards efficient comparison of change-based models,'' B. Combemale and A. Shaukat, Eds., vol. 18, no. 2, Jul. 2019, pp. 7:1–21, the 15th European Conference on Modelling Foundations and Applications. [Online]. Available: 
  \url{http://www.jot.fm/contents/issue_2019_02/article7.html} (Chapter 6).
\end{enumerate}
