\chapter{Literature Review}
This work has conducted a literature review and existing tool exploration to 
identify problem that provide motivation for research. 

\section{Identifying Changes in Models}
\label{sec:identifying_changes_in models}
There are two approaches in the literature for identifying changes in models 
in order to enable incremental re-execution of model processing operations.

\textbf{Notifications}. In this approach, the incremental execution 
engine needs to hook into the notification facilities 
provided by the modelling tool through which the developer edits the model, 
so that the engine can directly receive notifications as soon as 
changes happen (e.g. a new employee (\emph{e4}) has been added, 
the name property of employee \emph{e4} has been changed to ``Richmond"). 
This is an approach taken by the IncQuery incremental pattern matching 
framework \cite{DBLP:conf/ecmdafa/RathHV12} and the ReactiveATL incremental model-to-model 
transformation engine \cite{DBLP:conf/ecmdafa/OgunyomiRK15}. The main advantage of this 
approach is that precise and fine-grained change notifications are provided 
for free by the modelling tool (and thus do not need to be computed by the 
execution engine---which as discussed below can be expensive and inefficient). 
On the downside, this approach is a poor fit for collaborative development 
settings where modelling and automated model processing activities are 
performed by different members of the team.

\textbf{Model Differencing}. This approach eliminates the coupling between 
modelling tools and incremental execution engines. Instead of depending on 
live notifications, in this approach the developer in charge of automated model 
processing, needs to have access to a copy of the last version of the model that 
the model processing program (e.g. the model-to-text transformation) was 
executed upon, so that it can be compared against the current version of 
the model (e.g. using a model-differencing framework such as SiDiff or 
EMFCompare) and the delta can be computed on demand. The main advantage of 
this approach is that it works well in a collaborative development environment 
where typically developers have distinct roles and responsibilities. On the 
downside, model comparison and differencing are computationally expensive and 
memory-greedy (both versions of the model need to be loaded into memory before 
they can be compared), thus largely undermining the time and resource saving 
potentials of incremental re-execution. This approach is adopted by the Xpand 
model-to-text transformation language. According to the developers of the 
language, using this approach, a speed-up of only around 50\% is observed 
compared to non-incremental transformation\footnote{\url{http://wiki.eclipse.org/Xpand/
        New_And_Noteworthy\#Incremental_Generation}}, 
which is consistent with our experience from using Xpand.

In summary, incremental model processing currently delivers significant 
performance benefits only in a single-developer environment where the modeller 
is also responsible for performing all the (incremental) model processing 
operations. As a result, in collaborative development environments, 
developers need to either forgo incremental model processing altogether 
or to work around this limitation by manually steering model processing 
programs to process only subsets of their models, which is cumbersome and 
error prone.


\section{State-based Model vs. Change-based Model}
\label{sec:state-based_model_vs_change-based_model}
 The summary of 
the literature review and existing tool exploration is presented in
Table \ref{table:summary_literature_review}.

\begin{table}[t!]
    \centering
    \caption{Summary of this work's literature review on change-based approach.}
    \label{table:summary_literature_review}
    \begin{tabular}
        {|>{\centering\arraybackslash}p{2cm}|>{\centering\arraybackslash}p{1.6cm}|>{\centering\arraybackslash}p{4.7cm}|>{\centering\arraybackslash}p{4.7cm}|}
        \hline 
        \multicolumn{2}{|c|}{\textbf{Dimensions}}&\textbf{Change-based Approach}&\textbf{State-based Approach}\\
        \hline 
        \multirow{2}{2cm}{\centering Model change-detection} & methods & use notification to capture changes & compute differences between two models \\
        \hhline{~---}
        & products & used in many incremental projects (e.g IncQuery \cite{rath2012derived}, ReactiveATL \cite{ogunyomi2015property}) & SiDiff \cite{kelter2005generic}, EMF Compare \cite{eclipse2017compare}  \\ 
        \hline
        \multirow{2}{2cm}{\centering Model persistence} & methods & persist changes of models & persist states of models \\
        \hhline{~---}
        & products & EMFStore \cite{koegel2010emfstore} & XMI, NeoEMF \cite{daniel2016neoemf}, Morsa \cite{pagan2011morsa}, CDO \cite{eclipse2017cdo}, Teneo \cite{eclipse2017teneo}\\
        \hline
        \multirow{2}{2cm}{\centering Other persistence} & entities & software \cite{DBLP:journals/entcs/RobbesL07}, database \cite{DBLP:conf/sde/LippeO92}, hierarchical documents \cite{DBLP:conf/caise/IgnatN05}, model repository and version control \cite{koegel2010emfstore} & default for most persistence entities \\
        \hline
        \multicolumn{2}{|p{3.6cm}|}{\centering Advantages} &
        \begin{minipage}[t]{4.7cm}
            \raggedright
            \begin{itemize}[leftmargin=9pt]
                \setlength\itemsep{-5pt}
                \item[-] Faster for detecting changes \cite{DBLP:conf/edoc/KoegelHLHD10}
                \item[-] More accurate, carry semantic information \cite{DBLP:journals/entcs/RobbesL07,DBLP:conf/sde/LippeO92,DBLP:conf/caise/IgnatN05,mens2002state}  
                \item[-] Faster and more accurate for comparison and merging \cite{DBLP:conf/sde/LippeO92,DBLP:conf/caise/IgnatN05,koegel2010emfstore}
                \item[-] Information carried is useful for analytics \cite{DBLP:journals/entcs/RobbesL07}
            \end{itemize}
        \end{minipage}
        & 
        \begin{minipage}[t]{4.7cm}
            \raggedright
            \begin{itemize}[leftmargin=9pt]
                \setlength\itemsep{-5pt}
                \item[-] Faster for loading large models \cite{pagan2011morsa,daniel2016neoemf}
                \item[-] Supported by standard version controls (e.g. GitHub) \cite{koegel2010emfstore} 
                \item[-] A default standard, no need integration with existing tools \cite{koegel2010emfstore}  
            \end{itemize}
        \end{minipage}
        \\
        \hline
        \multicolumn{2}{|p{3.6cm}|}{\centering Disadvantages} & \begin{minipage}[t]{4.7cm}
            \raggedright
            \begin{itemize}[leftmargin=9pt]
                \setlength\itemsep{-5pt}
                \item[-] Increased record size \cite{DBLP:journals/entcs/RobbesL07,DBLP:conf/edoc/KoegelHLHD10}
                \item[-] Is not efficient for replaying (loading) for long records \cite{mens2002state}
                \item[-] Limited supports for standard, text-based version controls (e.g. GitHub) \cite{koegel2010emfstore} 
                \item[-] Not a standard, need integration with existing tools \cite{koegel2010emfstore} 
            \end{itemize}
        \end{minipage}
        & 
        \begin{minipage}[t]{4.7cm}
            \raggedright
            \begin{itemize}[leftmargin=9pt]
                \setlength\itemsep{-5pt}
                \item[-] Slower for saving changes  (currently NoSQL is used to solve this) \cite{mens2002state,daniel2016neoemf,pagan2011morsa}
                \item[-] Slower for comparison \cite{DBLP:conf/edoc/KoegelHLHD10}
                \item[-] Less accurate, does not carry semantic information \cite{mens2002state,DBLP:conf/edoc/KoegelHLHD10}  
            \end{itemize}
        \end{minipage}
        \\
        \hline
    \end{tabular} 
\end{table}

\section{EMF Compare}
\label{sec:emf_compare}

\section{EMF Store}
\label{sec:emf_store}


